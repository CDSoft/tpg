\section{Package content}

TPG is a package which main function is to take a grammar and return a parser\footnote{More precisely it returns the Python source code of the parser}.
You only need to import TPG and use these four objects:

\begin{description}
	\item [tpg.compile(grammar):]
		This function takes a grammar in a string and produces
		the Python object code of the parser.
		You can call exec to actually build it.
	\item [tpg.translate(grammar):]
		This function takes a grammar in a string and produces
		the Python source code of the parser.
		You can call exec to actually build it.
	\item [tpg.LexerError:]
		This exception is raised when the lexer fails.
	\item [tpg.ParserError:]
		This exception is raised when the parser fails.
	\item [tpg.SemanticError:]
		This exception is raised by the grammar itself
		when some semantic properties fail.
\end{description}

The grammar must be in a string (see figure~\ref{usage:embed}).

\begin{code}
\caption{Grammar embeding example}							\label{usage:embed}
\begin{verbatimtab}[4]
	my_grammar = r"""

	parser Foo:

		START/x -> Bar/x .

		Bar/x -> 'bar'/x .

	"""
\end{verbatimtab}
\end{code}

The \emph{tpg.compile} function produces Python code from the grammar (see figure~\ref{usage:comp}).

\begin{code}
\caption{Parser compilation example}						\label{usage:comp}
\begin{verbatimtab}[4]
	exec(tpg.compile(my_grammar))    # Compiles my_grammar
\end{verbatimtab}
\end{code}

Then you can use the new generated parser. The parser is now simply a Python class (see figure~\ref{usage:inst}).

\begin{code}
\caption{Parser usage example}								\label{usage:inst}
\begin{verbatimtab}[4]
	test = "bar"
	my_parser = Foo()
	x = my_parser(test)               # Uses the START symbol
	print x
	x = my_parser.parse('Bar', test)  # Uses the Bar symbol
	print x
\end{verbatimtab}
\end{code}

\section{Command line usage}

The \emph{tpg} script is just a wrapper for the package.
It reads a grammar in a file and write the generated code in a Python script.
To produce a Python script from a grammar you can use \emph{tpg} as follow:

\begin{verbatimtab}[4]
	tpg [-v|-vv] grammar.g [-o parser.py]
\end{verbatimtab}

\emph{tpg} accepts some options on the command line:

\begin{description}
	\item [-v] turns \emph{tpg} into a verbose mode (it displays parser names).
	\item [-vv] turns \emph{tpg} into a more verbose mode (it displays parser names and simplified rules).
	\item [-o file.py] tells \emph{tpg} to generate the parser in \emph{file.py}. The default output file is \emph{grammar.py} if -o option is not provided and \emph{grammar.g} is the name of the grammar.
\end{description}
