
\section{Getting TPG}

TPG is freely available on its web page (\url{http://christophe.delord.free.fr/en/tpg}). It is distributed as a package using \emph{distutils}\footnote{distutils is a Python package used to distribute Python softwares}.

\section{Requirements}

TPG is a \emph{pure Python} package.
It may run on \emph{any platform} supported by Python.
The only requirement of TPG is \emph{Python~2.2} or newer. Python can be downloaded at \url{http://www.python.org}.

\section{TPG for Linux and other Unix like}
\label{linux_install}

Download \mbox{\emph{TPG-X.Y.Z.tar.gz}}, unpack and run the installation program:
\begin{quote}
\begin{verbatim}
	tar xzf TPG-X.Y.Z.tar.gz
	cd TPG-X.Y.Z
	python setup.py install
\end{verbatim}
\end{quote}
You may need to be logged as root to install TPG.

\section{TPG for M\$ Windows}

Download \mbox{\emph{TPG-X.Y.Z.win32.exe}} and run it.

\section{TPG for other operating systems}

TPG should run on any system provided that Python is installed. You should be able to install it by running the \mbox{\emph{setup.py}} script (see \ref{linux_install}).
