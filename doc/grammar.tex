\section{TPG grammar structure}

TPG grammars may contain three parts:

\begin{description}
	\item [Options]
		are defined at the beginning of the grammar (see~\ref{grammar:options}).
	\item [Parsers]
		are described in sections starting with the \emph{parser} keyword (see~\ref{grammar:parser}).
	\item [Python codes]
		can appear in sections starting with the \emph{main} keyword or before the first parser (see~\ref{grammar:code}).
\end{description}

See figure~\ref{grammar:struct} for a generic TPG grammar.

\begin{code}
\caption{TPG grammar structure}								\label{grammar:struct}
\begin{verbatimtab}[4]
# Options
set magic = "/usr/bin/env python"

# Python code
{{
	class MyClass:
		pass
}}

# Parser Foo
parser Foo:

	START -> X Y Z ;

# More Python code
main:
{{
	def myfunction:
		pass
}}
\end{verbatimtab}
\end{code}

\section{Comments}

Comments in TPG start with \emph{\#} and run until the end of the line.

\begin{verbatimtab}[4]
	# This is a comment
\end{verbatimtab}

\section{Options}											\label{grammar:options}

Some options can be set at the beginning of TPG grammars (global options)
or at the beginning of each parser (local options).
The syntax for options is:

\begin{description}
	\item [set \emph{name}] sets the boolean \emph{name} option to \emph{true}.
	\item [set \emph{name} = "\emph{$value$}"] sets the \emph{name} option to \emph{$value$}.
	\item [set \emph{name} = "\emph{$value_1$}", ..., "\emph{$value_n$}"] sets the \emph{name} option to the list \emph{$value_1$}, ..., \emph{$value_n$} when more than one value in requested.
\end{description}

\subsection{Global options}

Global options are defined at the beginning of the file and are active for all parsers.

\subsubsection{Magic option}

The magic option tells TPG which interpreter is called when the script is run.
The first line of the generated code will start with \verb$#!$ and contains the command line to execute the appropriate interpreter (\verb$/usr/bin/env python$ for example).
This has no effect on M\$ Windows.

\begin{description}
	\item [set magic = "/usr/bin/env python"] adds \verb$#!/usr/bin/env python$ to the first line.
\end{description}

\subsection{Local options}

Local options are defined at the beginning of each parser and only apply to the parser in which they are defined.

\subsubsection{CSL option}									\label{grammar:CSL}

By default TPG lexers are context free.
The \emph{CSL} option tells TPG to generate a context sensitive lexer (see~\ref{tpg:CSL}).

\begin{description}
	\item [set CSL] generates context sensitive lexers.
\end{description}

\subsubsection{Indent option}                               \label{grammar:indent_option}

This option adds \emph{indent} and \emph{deindent} tokens to the current parser.
These tokens are similar to INDENT and DEINDENT tokens in Python language.

\begin{description}
    \item [set indent = "indent\_re", "no\_indent\_re"] tells TPG how to recognize indentation.
        \emph{indent\_re} is a regular expression for indentation (usually spaces and tabulations).
        \emph{no\_indent\_re} is a regular expression for lines to be ignores (usually comments).
\end{description}

\section{Python code}										\label{grammar:code}

Python code section are not handled by TPG.
TPG won't complain about syntax errors in Python code sections, it is Python's job.
They are copied verbatim to the generated Python parser.
The first piece of code written before the first parser is generated at the very beginning of the parser Code.
This is where you can put \emph{from \_\_future\_\_ import ...} statements for instance.

\subsection{Syntax}

Python code is enclosed in double curly brackets.
That means that Python code must not contain to consecutive close brackets.
You can avoid this by writting \emph{\}~\}} (with a space) instead of \emph{\}\}} (without space).

\subsection{Indentation}

Python code can appear in several parts of a grammar.
Since indentation has a special meaning in Python it is important to know how TPG handles spaces and tabulations at the beginning of the lines.
In TPG indentation is important only in Python code sections (in \emph{main} parts, in \emph{parser} parts and in \emph{rules}).

When TPG encounters some Python code it removes in all non blank lines the spaces and tabulations that are common to every lines.
TPG considers spaces and tabulations as the same character so it is important to always use the same indentation style.
Thus it is advised not to mix spaces and tabulations in indentation.
Then this code will be reindented when generated according to its location (in a class, in a method or in global space).

The figure~\ref{grammar:indent} shows how TPG handles indentation.

\begin{tableau}
\caption{Code indentation examples}							\label{grammar:indent}
\begin{tabular}{| p{5cm} | p{4cm} | p{4cm} |}
\hline
	Code in grammars & Generated code & Comment \\
\hline
\hline

	\begin{verbatim*}
{{
    if 1==2:
        print "???"
    else:
        print "OK"
}}
	\end{verbatim*}
	&
	\begin{verbatim*}

if 1==2:
    print "???"
else:
    print "OK"
	\end{verbatim*}
	&
	\emph{Correct}: these lines have four spaces in common.  These spaces are removed.
	\\

\hline

	\begin{verbatim*}
{{  if 1==2:
        print "???"
    else:
        print "OK"
}}
	\end{verbatim*}
	&
	\begin{verbatim*}
if 1==2:
      print "???"
  else:
      print "OK"
	\end{verbatim*}
	&
	\emph{WRONG}: it's a bad idea to start a multiline code section on the first line since the common indentation may be different from what you expect.  No error will be raised by TPG but Python won't compile this code.
	\\

\hline

	\begin{verbatim*}
{{       print "OK" }}
	\end{verbatim*}
	&
	\begin{verbatim*}
print "OK"
	\end{verbatim*}
	&
	\emph{Correct}: indentation does not matter in a one line Python code.
	\\

\hline

\end{tabular}
\end{tableau}

\section{TPG parsers}										\label{grammar:parser}

A grammar can contain as many parsers as needed.
A parser declaration starts with the \emph{parser} keyword and contains rules and Python code sections (local to the parser).

\subsection{Initialisation}

The initialisation of Python objects is made by the \emph{\_\_init\_\_} method.
This method is generated by TPG and cannot be overriden.
To resolve this problem an \emph{init} method (i.e. without the double underscores) is called at initialization time with the arguments given to \emph{\_\_init\_\_}.
See~\ref{grammar:python_code} to add methods to a parser.

\subsection{Rules}

Each rule will be translated into a method of the parser.

\subsection{Python code}									\label{grammar:python_code}

Python code that is local to a parser will be copied in the generated class.
This is usually used to add methods or attributes to the parser.
