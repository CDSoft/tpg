% HTML: htlatex tpg html,2          1 time
% DVI : latex tpg                   3 times
% PDF : pdflatex tpg                3 times

%\documentclass[a4paper,twoside,11pt]{report}
\documentclass[a4paper,twoside]{report}

\usepackage{a4wide, moreverb}

\usepackage{ifpdf}
%\newif\ifpdf
%\ifx\pdfoutput\undefined
%    \pdffalse
%\else
%    \pdfoutput=1
%    \pdftrue
%\fi

\ifpdf
    \usepackage[
        pdftex,
        pdftitle={Toy Parser Generator},
        pdfsubject={A simple parser generator for Python},
        pdfcreator={Christophe Delord <cdsoft.fr>},
        pdfauthor={Christophe Delord <cdsoft.fr>},
        pdfproducer={Christophe Delord <cdsoft.fr>},
        pdfkeywords={Python, Parser, Generator, Toy Parser Generator},
        colorlinks=true
    ]{hyperref}
    \pdfcompresslevel=9
\else
    \usepackage[
        pdftitle={Toy Parser Generator},
        pdfsubject={A simple parser generator for Python},
        pdfcreator={Christophe Delord <cdsoft.fr>},
        pdfauthor={Christophe Delord <cdsoft.fr>},
        pdfproducer={Christophe Delord <cdsoft.fr>},
        pdfkeywords={Python, Parser, Generator, Toy Parser Generator},
        colorlinks=true
    ]{hyperref}
\fi

\renewcommand{\url}[1]{\href{#1}{#1}}

\author{
    Christophe~Delord\\
    \url{http://cdsoft.fr/tpg/} \\
}
\title{
    Toy~Parser~Generator \\
    or\\
    How to easily write parsers in Python \\
}

\pagestyle{headings}

\newenvironment{tableau}
{
    \begin{figure}[ht]
    \begin{center}
}
{
    \end{center}
    \end{figure}
}

\newenvironment{code}
{
    \begin{figure}[ht]
    \begin{tabular}{| p{\textwidth} |}
    \hline
}
{
    \\
    \hline
    \end{tabular}
    \end{figure}
}

\begin{document}

\maketitle
\tableofcontents

\listoffigures
%\listoftables

\cleardoublepage
\part{Introduction and tutorial}                            \label{tpg:intro}
    \chapter{Introduction}
        \section{Introduction}

TPG (Toy Parser Generator) is a Python\footnote{Python is a wonderful object oriented programming language available at \url{http://www.python.org}} parser generator.
It is aimed at easy usage rather than performance.
My inspiration was drawn from two different sources.
The first was GEN6. GEN6 is a parser generator created at ENSEEIHT\footnote{ENSEEIHT is a french engineer school (\url{http://www.enseeiht.fr}).} where I studied.
The second was PROLOG\footnote{PROLOG is a programming language using logic. My favorite PROLOG compiler is SWI-PROLOG (\url{http://www.swi-prolog.org}).}, especially DCG\footnote{Definite Clause Grammars.} parsers.
I wanted a generator with a simple and expressive syntax and the generated parser should work as the user expects. So I decided that TPG should be a recursive descendant parser (a rule is a procedure that calls other procedures) and the grammars are attributed (attributes are the parameters of the procedures).
This way TPG can be considered as a programming language or more modestly as Python extension.

\section{License}

TPG is available under the GNU Lesser General Public.

\begin{quote}
Toy Parser Generator: A Python parser generator

Copyright (C) 2002 Christophe Delord
 
This library is free software; you can redistribute it and/or
modify it under the terms of the GNU Lesser General Public
License as published by the Free Software Foundation; either
version 2.1 of the License, or (at your option) any later version.

This library is distributed in the hope that it will be useful,
but WITHOUT ANY WARRANTY; without even the implied warranty of
MERCHANTABILITY or FITNESS FOR A PARTICULAR PURPOSE.  See the GNU
Lesser General Public License for more details.

You should have received a copy of the GNU Lesser General Public
License along with this library; if not, write to the Free Software
Foundation, Inc., 59 Temple Place, Suite 330, Boston, MA  02111-1307  USA 
\end{quote}

\section{Structure of the document}

\begin{description}
\item [Part~\ref{tpg:intro}]
starts smoothly with a gentle tutorial as an introduction.
I think this tutorial may be sufficent to start with TPG.
\item [Part~\ref{tpg:core}]
is a reference documentation. It will detail TPG as much as possible.
\item [Part~\ref{tpg:examples}]
gives the reader some examples to illustrate TPG.
\item [Part~\ref{tpg:hack}]
is an explanation of how TPG works internally. It details the predictive algorithm and shows the generated code. It is not needed to read this part but it can help to understand how TPG works or why some grammars fail.
\end{description}

    \chapter{Installation}
        
\section{Getting TPG}

TPG is freely available on its web page (\url{http://christophe.delord.free.fr/en/tpg}). It is distributed as a package using \emph{distutils}\footnote{distutils is a Python package used to distribute Python softwares}.

\section{Requirements}

TPG is a \emph{pure Python} package.
It may run on \emph{any platform} supported by Python.
The only requirement of TPG is \emph{Python~2.2} or newer. Python can be downloaded at \url{http://www.python.org}.

\section{TPG for Linux and other Unix like}					\label{linux_install}

Download \mbox{\emph{TPG-X.Y.Z.tar.gz}}, unpack and run the installation program:
\begin{quote}
\begin{verbatim}
	tar xzf TPG-X.Y.Z.tar.gz
	cd TPG-X.Y.Z
	python setup.py install
\end{verbatim}
\end{quote}
You may need to be logged as root to install TPG.

\section{TPG for M\$ Windows}

Download \mbox{\emph{TPG-X.Y.Z.win32.exe}} and run it.

\section{TPG for other operating systems}

TPG should run on any system provided that Python is installed. You should be able to install it by running the \mbox{\emph{setup.py}} script (see~\ref{linux_install}).

    \chapter{Tutorial}                                      \label{tpg:tutorial}
        \section{Introduction}

This short tutorial presents how to make a simple calculator.
The calculator will compute basic mathematical expressions (\verb|+, -, *, /|) possibly nested in parenthesis.
We assume the reader is familiar with regular expressions.

\section{Defining the grammar}

Expressions are defined with a grammar.
For example an expression is a term, a term is a sum of factors and a factor is a product of atomic expressions. An atomic expression is either a number or a complete expression in parenthesis.

We describe such grammars with rules. A rule describe the composition of an item of the language. In our grammar we have 3 items (Term, Factor, Atom). We will call these items `symbols' or `non terminal symbols'. The decomposition of a symbol is symbolized with $\to$.
The grammar of this tutorial is given in figure~\ref{tut:gramsymb_calc}.

\begin{tableau}
\caption{Grammar for expressions}							\label{tut:gramsymb_calc}
\begin{tabular}{| l | p{7cm} |}
\hline
	Grammar rule & Description \\
\hline
\hline
	$Term~\to~Factor~(('+'|'-')~Factor)*$
		& A term is a factor eventually followed with a plus ($'+'$) or a minus ($'-'$) sign and an other factor any number of times ($*$ is a repetition of an expression 0 or more times). \\
\hline
	$Factor~\to~Atom~(('*'|'/')~Atom)*$
		& A factor is an atom eventually followed with a $'*'$ or $'/'$ sign and an other atom any number of times. \\
\hline
	$Atom~\to~number~|~'('~Term~')'$
		& An atomic expression is either a number or a term in parenthesis. \\
\hline
\end{tabular}
\end{tableau}

We have defined here the grammar rules (i.e. the sentences of the language). We now need to describe the lexical items (i.e. the words of the language). These words - also called \emph{terminal symbols} - are described using regular expressions. In the rules we have written some of these terminal symbols ($+, -, *, /, (, )$). We have to define \emph{number}. For sake of simplicity numbers are integers composed of digits (the corresponding regular expression can be $[0-9]+$).
To simplify the grammar and then the Python script we define two terminal symbols to group the operators (additive and multiplicative operators). We can also define a special symbol that is ignored by TPG. This symbol is used as a separator. This is generaly usefull for white spaces and comments. The terminal symbols are given in figure~\ref{tut:token_calc}

\begin{tableau}
\caption{Terminal symbol definition for expressions}		\label{tut:token_calc}
\begin{tabular}{| l | l | l |}
\hline
	Terminal symbol & Regular expression & Comment \\
\hline
\hline
	number & $[0-9]+$ or $\backslash d+$ & One or more digits \\
\hline
	add & $[+-]$ & a $+$ or a $-$ \\
\hline
	mul & $[*/]$ & a $*$ or a $/$ \\
\hline
	spaces & $\backslash s+$ & One or more spaces \\
\hline
\end{tabular}
\end{tableau}

This is sufficient to define our parser with TPG. The grammar of the expressions in TPG can be found in figure~\ref{tut:recognizer}.

\begin{code}
\caption{Grammar of the expression recognizer}				\label{tut:recognizer}
\begin{verbatimtab}[4]
parser Calc:

	separator spaces: '\s+' ;
	token number: '\d+' ;
	token add: '[+-]' ;
	token mul: '[*/]' ;

	START -> Term ;

	Term -> Fact ( add Fact )* ;

	Fact -> Atom ( mul Atom )* ;

	Atom -> number | '\(' Term '\)' ;
\end{verbatimtab}
\end{code}

\emph{Calc} is the name of the Python class generated by TPG. \emph{START} is a special non terminal symbol treated as the \emph{axiom}\footnote{The axiom is the symbol from which the parsing starts} of the grammar.

With this small grammar we can only recognize a correct expression. We will see in the next sections how to read the actual expression and to compute its value.

\section{Reading the input and returning values}

The input of the grammar is a string. To do something useful we need to read this string in order to transform it into an expected result.

This string can be read by catching the return value of terminal symbols. By default any terminal symbol returns a string containing the current token. So the token $'('$ always returns the string $'('$. For some tokens it may be useful to compute a Python object from the token. For example \emph{number} should return an integer instead of a string, \emph{add} and \emph{mul} should return a function corresponding to the operator. That why we will add a function to the token definitions. So we associate \emph{int} to \emph{number} and \emph{make\_op} to \emph{add} and \emph{mul}.

\emph{int} is a Python function converting objects to integers and \emph{make\_op} is a user defined function (figure~\ref{tut:make_op}).

\begin{code}
\caption{\emph{make\_op} function}							\label{tut:make_op}
\begin{verbatimtab}[4]
def make_op(s):
	return {
		'+': lambda x,y: x+y,
		'-': lambda x,y: x-y,
		'*': lambda x,y: x*y,
		'/': lambda x,y: x/y,
	}[s]
\end{verbatimtab}
\end{code}

To associate a function to a token it must be added after the token definition as in figure~\ref{tut:tokens}

\begin{code}
\caption{Token definitions with functions}					\label{tut:tokens}
\begin{verbatimtab}[4]
	separator spaces: '\s+' ;
	token number: '\d+' int ;
	token add: '[+-]' make_op;
	token mul: '[*/]' make_op;
\end{verbatimtab}
\end{code}

We have specified the value returned by the token. To read this value after a terminal symbol is recognized we will store it in a Python variable. For example to save a \emph{number} in a variable \emph{n} we write \emph{number/n}.
In fact terminal and non terminal symbols can return a value. The syntax is the same for both sort of symbols. In non terminal symbol definitions the return value defined at the left hand side is the expression return by the symbol. The return values defined in the right hand side are just variables to which values are saved. A small example may be easier to understand (figure~\ref{tut:ret_val}).

\begin{tableau}
\caption{Return values for (non) terminal symbols}			\label{tut:ret_val}
\begin{tabular}{| l | p{9cm} |}
\hline
	Rule & Comment \\
\hline
	\verb!X/x ->!			& Defines a symbol \emph{X}. When \emph{X} is called, \emph{x} is returned. \\
	\verb!Y/y!				& \emph{X} starts with a \emph{Y}. The return value of \emph{Y} is saved in \emph{y}. \\
	\verb!Z/z!				& The return value of \emph{Z} is saved in \emph{z}. \\
	\verb!{{ x = y+z }}!	& Computes \emph{x}. \\
	\verb!;!				& Returns \emph{x}. \\
\hline
\end{tabular}
\end{tableau}

In the example described in this tutorial the computation of a \emph{Term} is made by applying the operator to the factors, this value is then returned :

\begin{verbatimtab}[4]
	Term/t -> Fact/t ( add/op Fact/f {{ t = op(t,f) }} )* ;
\end{verbatimtab}

This example shows how to include Python code in a rule. Here \verb!{{...}}! is copied verbatim in the generated parser.

Finally the complete parser is given in figure~\ref{tut:parser}.

\begin{code}
\caption{Expression recognizer and evaluator}				\label{tut:parser}
\begin{verbatimtab}[4]
parser Calc:

	separator spaces: '\s+' ;

	token number: '\d+' int ;
	token add: '[+-]' make_op ;
	token mul: '[*/]' make_op ;

	START -> Term ;

	Term/t -> Fact/t ( add/op Fact/f {{ t = op(t,f) }} )* ;

	Fact/f -> Atom/f ( mul Atom/a {{ f = op(f,a) }} )* ;

	Atom/a -> number/a | '\(' Term/a '\)' ;
\end{verbatimtab}
\end{code}

\section{Embeding the parser in a script}

To embed a TPG parser in a Python program, you only need the \emph{tpg.compile} function.
This function takes a grammar (in a string\footnote{It may be a good pratice to use only raw strings. This will ease the pain of writing regular expressions.}) and returns a string containing the Python code for the parser.
One way to use this parser is to \emph{exec} its definition.
A practical way to build parsers is to \emph{exec} the result of \emph{tpg.compile} (figure~\ref{tut:build_scheme}).

\begin{code}
\caption{Python code generation from a grammar}				\label{tut:build_scheme}
\begin{verbatimtab}[4]
import tpg

exec(tpg.compile(r""" # Your grammar here """))

# You can instanciate your parser here
\end{verbatimtab}
\end{code}

To use this parser you now just need to import tpg, compile the grammar and instanciate an object of the class \emph{Calc} as in figure~\ref{tut:calc}.

\begin{code}
\caption{Complete Python script with expression parser}		\label{tut:calc}
\begin{verbatimtab}[4]
import tpg

def make_op(s):
	return {
		'+': lambda x,y: x+y,
		'-': lambda x,y: x-y,
		'*': lambda x,y: x*y,
		'/': lambda x,y: x/y,
	}[s]

exec(tpg.compile(r"""

parser Calc:

	separator spaces: '\s+' ;

	token number: '\d+' int ;
	token add: '[+-]' make_op ;
	token mul: '[*/]' make_op ;

	START/e -> Term/e ;
	Term/t -> Fact/t ( add/op Fact/f {{ t = op(t,f) }} )* ;
	Fact/f -> Atom/f ( mul/op Atom/a {{ f = op(f,a) }} )* ;
	Atom/a -> number/a | '\(' Term/a '\)' ;

"""))

calc = Calc()
expr = raw_input('Enter an expression: ')
print expr, '=', calc(expr)
\end{verbatimtab}
\end{code}

\clearpage

\section{Conclusion}

This tutorial shows some of the possibilities of TPG.
If you have read it carefully you may be able to start with TPG.
The next chapters present TPG more precisely.
They contain more examples to illustrate all the features of TPG.

Happy TPG'ing!


\cleardoublepage
\part{TPG reference}                                        \label{tpg:core}
    \chapter{Usage}
        \section{Package content}

TPG is a package which main function is to take a grammar and return a parser\footnote{More precisely it returns the Python source code of the parser}.
You only need to import TPG and use these four objects:

\begin{description}
	\item [tpg.compile(grammar):]
		This function takes a grammar in a string and produces
		a parser in Python (also in a string).
		You can call exec to actually build it. 
	\item [tpg.LexicalError:]
		This exception is raised when the lexer fails.
	\item [tpg.SyntaxError:]
		This exception is raised when the parser fails.
	\item [tpg.SemanticError:]
		This exception is raised by the grammar itself
		when some semantic properties fail.
\end{description}

The grammar must be in a string (see figure~\ref{usage:embed}).

\begin{code}
\caption{Grammar embeding example}							\label{usage:embed}
\begin{verbatimtab}[4]
	my_grammar = r"""

	parser Foo:

		START/x -> Bar/x .

		Bar/x -> 'bar'/x .

	"""
\end{verbatimtab}
\end{code}

The \emph{tpg.compile} function produces Python code from the grammar (see figure~\ref{usage:comp}).

\begin{code}
\caption{Parser compilation example}						\label{usage:comp}
\begin{verbatimtab}[4]
	exec(tpg.compile(my_grammar))    # Compiles my_grammar
\end{verbatimtab}
\end{code}

Then you can use the new generated parser. The parser is now simply a Python class (see figure~\ref{usage:inst}).

\begin{code}
\caption{Parser usage example}								\label{usage:inst}
\begin{verbatimtab}[4]
	test = "bar"
	my_parser = Foo()
	x = my_parser(test)               # Uses the START symbol
	print x
	x = my_parser.parse('Bar', test)  # Uses the Bar symbol
	print x
\end{verbatimtab}
\end{code}

\section{Command line usage}

The \emph{tpg} script is just a wrapper for the package.
It reads a grammar in a file and write the generated code in a Python script.
To produce a Python script from a grammar you can use \emph{tpg} as follow:

\begin{verbatimtab}[4]
	tpg [-v|-vv] grammar.g [-o parser.py]
\end{verbatimtab}

\emph{tpg} accepts some options on the command line:

\begin{description}
	\item [-v] turns \emph{tpg} into a verbose mode (it displays parser names).
	\item [-vv] turns \emph{tpg} into a more verbose mode (it displays parser names and simplified rules).
	\item [-o file.py] tells \emph{tpg} to generate the parser in \emph{file.py}. The default output file is \emph{grammar.py} if -o option is not provided and \emph{grammar.g} is the name of the grammar.
\end{description}

    \chapter{Grammar structure}
        \section{TPG grammar structure}

TPG grammars are contained in the doc string\footnote{If the grammar (i.e. the doc string) is a unicode string then the generated parser can parse unicode strings} of the parser class.
TPG grammars may contain three parts:

\begin{description}
    \item [Options]
        are defined at the beginning of the grammar (see~\ref{grammar:options}).
    \item [Tokens]
        are introduced by the \emph{token} or \emph{separator} keyword (see~\ref{lexer:token_def}).
    \item [Rules]
        are described after tokens (see~\ref{grammar:parser}).
\end{description}

See figure~\ref{grammar:struct} for a generic TPG grammar.

\begin{code}
\caption{TPG grammar structure}                             \label{grammar:struct}
\begin{verbatimtab}[4]
class Foo(tpg.Parser):
    r"""

        # Options
        set lexer = CSL

        # Tokens
        separator spaces    '\s+'       ;
        token int           '\d+'   int ;

        # Rules
        START -> X Y Z ;

    """

foo = Foo()
result = foo("input string")
\end{verbatimtab}
\end{code}

\section{Comments}

Comments in TPG start with \emph{\#} and run until the end of the line.

\begin{verbatimtab}[4]
    # This is a comment
\end{verbatimtab}

\section{Options}                                           \label{grammar:options}

Some options can be set at the beginning of TPG grammars.
The syntax for options is:

\begin{description}
    \item [set \emph{name} = \emph{$value$}] sets the \emph{name} option to \emph{$value$}.
\end{description}

\subsection{Lexer option}                                   \label{grammar:lexer_option}

The \emph{lexer} option tells TPG which lexer to use.

\begin{description}
    \item [set lexer = NamedGroupLexer] is the default lexer.
        It is context free and uses named groups of the \emph{sre} package (and its limitation of 100 named groups, ie 100 tokens).
    \item [set lexer = Lexer] is similar to \emph{NamedGroupLexer} but doesn't use named groups.
        It is slower than \emph{NamedGroupLexer}.
    \item [set lexer = CacheNamedGroupLexer] is similar to \emph{NamedGroupLexer} except that tokens are first stored in a list.
        It is faster for heavy backtracking grammars.
    \item [set lexer = CacheLexer] is similar to \emph{Lexer} except that tokens are first stored in a list.
        It is faster for heavy backtracking grammar.
    \item [set lexer = ContextSensitiveLexer] is the context sensitive lexer (see~\ref{tpg:CSL}).
\end{description}

\subsection{Word bondary option}                            \label{grammar:word_boundary_option}

The \emph{word\_boundary} options tells the lexer to search for word boundaries after identifiers.

\begin{description}
    \item [set word\_boundary = True] enables the word boundary search. This is the default.
    \item [set word\_boundary = False] disables the word boundary search.
\end{description}

\subsection{Regular expression options}

The \emph{sre} module accepts some options to define the behaviour of the compiled regular expressions.
These options can be changed for each parser.

\begin{description}
    \item [set lexer\_ignorecase = True] enables the \emph{re.IGNORECASE} option.
    \item [set lexer\_locale = True] enables the \emph{re.LOCALE} option.
    \item [set lexer\_multiline = True] enables the \emph{re.MULTILINE} option.
    \item [set lexer\_dotall = True] enables the \emph{re.DOTALL} option.
    \item [set lexer\_verbose = True] enables the \emph{re.VERBOSE} option.
    \item [set lexer\_unicode = True] enables the \emph{re.UNICODE} option.
\end{description}

\section{Python code}                                       \label{grammar:code}

Python code section are not handled by TPG.
TPG won't complain about syntax errors in Python code sections, it is Python's job.
They are copied verbatim to the generated Python parser.

\subsection{Syntax}

Before TPG 3, Python code is enclosed in double curly brackets.
That means that Python code must not contain to consecutive close brackets.
You can avoid this by writting \emph{\}~\}} (with a space) instead of \emph{\}\}} (without space).
This syntaxe is still available but the new syntax may be more readable.
The new syntax uses \emph{\$} to delimit code sections.
When several \emph{\$} sections are consecutive they are seen as a single section.

\subsection{Indentation}

Python code can appear in several parts of a grammar.
Since indentation has a special meaning in Python it is important to know how TPG handles spaces and tabulations at the beginning of the lines.

When TPG encounters some Python code it removes in all non blank lines the spaces and tabulations that are common to every lines.
TPG considers spaces and tabulations as the same character so it is important to always use the same indentation style.
Thus it is advised not to mix spaces and tabulations in indentation.
Then this code will be reindented when generated according to its location (in a class, in a method or in global space).

The figure~\ref{grammar:indent} shows how TPG handles indentation.

\begin{tableau}
\caption{Code indentation examples}                         \label{grammar:indent}
\begin{tabular}{| p{3.5cm} | p{3.5cm} | p{3cm} | p{3cm} |}
\hline
    Code in grammars (old syntax) & Code in grammars (new syntax) & Generated code & Comment \\
\hline
\hline

    \begin{verbatim*}
{{
    if 1==2:
        print "???"
    else:
        print "OK"
}}
    \end{verbatim*}
    &
    \begin{verbatim*}

$  if 1==2:
$      print "???"
$  else:
$      print "OK"

    \end{verbatim*}
    &
    \begin{verbatim*}

if 1==2:
    print "???"
else:
    print "OK"
    \end{verbatim*}
    &
    \emph{Correct}: these lines have four spaces in common.  These spaces are removed.
    \\

\hline

    \begin{verbatim*}
{{  if 1==2:
        print "???"
    else:
        print "OK"
}}
    \end{verbatim*}
    &
The new syntax has no trouble in that case.
    &
    \begin{verbatim*}
if 1==2:
      print "???"
  else:
      print "OK"
    \end{verbatim*}
    &
    \emph{WRONG}: it's a bad idea to start a multiline code section on the first line since the common indentation may be different from what you expect.  No error will be raised by TPG but Python won't compile this code.
    \\

\hline

    \begin{verbatim*}
{{    print "OK" }}
    \end{verbatim*}
    &
    \begin{verbatim*}
$       print "OK"
    \end{verbatim*}
or
    \begin{verbatim*}
$ print "OK" $
    \end{verbatim*}
    &
    \begin{verbatim*}
print "OK"
    \end{verbatim*}
    &
    \emph{Correct}: indentation does not matter in a one line Python code.
    \\

\hline

\end{tabular}
\end{tableau}

\section{TPG parsers}                                       \label{grammar:parser}

TPG parsers are \emph{tpg.Parser} classes.
The grammar is the doc string of the class.

\subsection{Methods}

As TPG parsers are just Python classes, you can use them as normal classes.
If you redefine the \emph{\_\_init\_\_} method, don't forget to call \emph{tpg.Parser.\_\_init\_\_}.

\subsection{Rules}

Each rule will be translated into a method of the parser.


    \chapter{Lexer}
        \section{Regular expression syntax}

The lexer is based on the \emph{re}\footnote{\emph{re} is a standard Python module. It handles regular expressions. For further information about \emph{re} you can read \url{http://python.org/doc/2.2/lib/module-re.html}} module.
TPG profits from the power of Python regular expressions.
This document assumes the reader is familiar with regular expressions.

You can use the syntax of regular expressions as expected by the \emph{re} module except from the grouping syntax since it is used by TPG to decide which token is recognized.

\section{Token definition}
\label{lexer:token_def}

\subsection{Predefined tokens}

Tokens can be explicitely defined by the \emph{token} and \emph{separator} keywords.

A token is defined by:

\begin{description}
	\item [a name] which identifies the token.
		This name is used by the parser.
	\item [a regular expression] which describes what to match to recognize the token.
	\item [an action] which can translate the matched text into a Python object. It can be a function of one argument or a non callable object. It it is not callable, it will be returned for each token otherwise it will be applied to the text of the token and the result will be returned. This action is optional. By default the token text is returned.
\end{description}

Token definitions end with a \emph{;} .

See figure~\ref{lexer:tokens} for examples.

\begin{code}
\caption{Token definition examples} \label{lexer:tokens}
\begin{verbatimtab}[4]
	#     name     reg. exp        action
	token integer: '\d+'           int;
	token ident  : '[a-zA-Z]\w*'   ;

	separator spaces  : '\s+';     # white spaces
	separator comments: '#.*';     # comments
\end{verbatimtab}
\end{code}

The order of the declaration of the tokens is important. The first token that is matched is returned. The regular expression has a special treatment. If it describes a keyword, TPG also looks for a word boundary after the keyword. If you try to match the keywords \emph{if} and \emph{ifxyz} TPG will internally search \verb$if\b$ and \verb$ifxyz\b$. This way, \emph{if} won't match \emph{ifxyz} and won't interfere with general identifiers (\verb$\w+$ for example).

There are two kinds of tokens. Tokens defined by the \emph{token} keyword are parsed by the parser and tokens defined by the \emph{separator} keyword are considered as separators (white spaces or comments for example) and are wiped out by the lexer.

\subsection{Inline tokens}

Tokens can also be defined on the fly. Their definition are then inlined in the grammar rules.
This feature may be useful for keywords or punctuation signs.
Inline tokens can not be transformed by an action as predefined tokens.
They always return the token in a string.

See figure~\ref{lexer:inline_tokens} for examples.

\begin{code}
\caption{Inline token definition examples} \label{lexer:inline_tokens}
\begin{verbatimtab}[4]
	IfThenElse ->
		'if' Cond
		'then' Statement
		'else' Statement
		;
\end{verbatimtab}
\end{code}

Inline tokens have a higher precedence than predefined tokens to avoid conflicts (an inlined \emph{if} won't be matched as a predefined \emph{identifier}).

\section{Token matching}
\label{lexer:token_matching}

TPG works in two stages.
The lexer first splits the input string into a list of tokens and then the parser parses this list.

\subsection{Splitting the input string}

The lexer split the input string according to the token definitions (see \ref{lexer:token_def}). When the input string can not be matched a \emph{tpg.LexicalError} exception is raised.

The lexer may loop indefinitely if a token can match an empty string since empty strings are everywhere.

\subsection{Matching tokens in grammar rules}

Tokens are matched as symbols are recognized.
Predefined tokens have the same syntax than non terminal symbols.
The token text (or the result of the function associated to the token) can be saved by the infix \emph(/) operator (see figure~\ref{lexer:token_ret_val}).

\begin{code}
\caption{Token usage examples} \label{lexer:token_ret_val}
\begin{verbatimtab}[4]
	S -> ident/i;
\end{verbatimtab}
\end{code}

Inline tokens have a similar syntax. You just write the regular expression (in a string). Its text can also be save (see figure~\ref{lexer:inline_token_ret_val}).

\begin{code}
\caption{Token usage examples} \label{lexer:inline_token_ret_val}
\begin{verbatimtab}[4]
	S -> '(' '\w+'/i ')';
\end{verbatimtab}
\end{code}

    \chapter{Parser}
        \section{Declaration}

A parser is declared as a \emph{tpg.Parser} class.
The doc string of the class contains the definition of the tokens and rules.

\section{Grammar rules}                                         \label{parser:grammar_rules}

Rule declarations have two parts.
The left side declares the symbol associated to the rule, its attributes and its return value.
The right side describes the decomposition of the rule.
Both parts of the declaration are separated with an arrow (\emph{$\to$})
and the declaration ends with a \emph{;}.

The symbol defined by the rule as well as the symbols that appear in the rule can have attributes and return values.
The attribute list - if any - is given as an object list enclosed in left and right angles.
The return value - if any - is extracted by the infix \emph{/} operator.
When no return value is specified, TPG creates a variable named as the symbol.
See figure~\ref{parser:rule} for example.

\begin{code}
\caption{Rule declaration}                                      \label{parser:rule}
\begin{verbatimtab}[4]
    SYMBOL <att1, att2, att3> / return_expression_of_SYMBOL ->

        A <x, y> / ret_value_of_A

        B <y, z> / ret_value_of_B

        ;

    S1 / $f(A,B)$ ->
        A       # return value of A stored in variable A
        B       # return value of B stored in variable B
    ;

    S2 ->       # we can use S2 to compute the return value
        A       $ S2 = A
        B       $ S2 = f(S2, B)
    ;
\end{verbatimtab}
\end{code}

\section{Parsing terminal symbols}

Each time a terminal symbol is encountered in a rule, the parser compares it to the current token in the token list. If it is different the parser backtracks.

\section{Parsing non terminal symbols}                          \label{parser:nterm}

\subsection{Starting the parser}

You can start the parser from the axiom or from any other non terminal symbol.
When the parser can not parse the whole token list a \emph{tpg.SyntacticError} is raised.
The value returned by the parser is the return value of the parsed symbol.

\subsubsection{From the axiom}

The axiom is a special non terminal symbol named \emph{START}.
Parsers are callable objects.
When an instance of a parser is called, the \emph{START} rule is parsed.
The first argument of the call is the string to parse.
The other arguments of the call are given to the \emph{START} symbol.

This allows to simply write \verb!x=calc("1+1")! to parse and compute an expression if \emph{calc} is an instance of an expression parser.

\subsubsection{From another non terminal symbol}

It's also possible to start parsing from any other non terminal symbol.
TPG parsers have a method named \emph{parse}.
The first argument is the name of the symbol to start from.
The second argument is the string to parse.
The other arguments are given to the specified symbol.

For example to start parsing a \emph{Term}
you can write:
\begin{verbatimtab}[4]
    f=calc.parse('Term', "2*3")
\end{verbatimtab}

\subsection{In a rule}

To parse a non terminal symbol in a rule, TPG call the rule corresponding to the symbol.

\section{Sequences}                                             \label{parser:sequences}

Sequences in grammar rules describe in which order symbols should appear in the input string.
For example the sequence \emph{A~B} recognizes an \emph{A} followed by a \emph{B}.
Sequences can be empty.

For example to say that a \emph{sum} is a \emph{term} \emph{plus} another \emph{term} you can write:
\begin{verbatimtab}[4]
    Sum -> Term '+' Term ;
\end{verbatimtab}

\section{Alternatives}                                          \label{parser:alternatives}

Alternatives in grammar rules describe several possible decompositions of a symbol.
The infix pipe operator (\emph{$\mid$}) is used to separate alternatives.
\emph{$A~\mid~B$} recognizes either an \emph{A} or a \emph{B}.
If both \emph{A} and \emph{B} can be matched only the first match is considered.
So the order of alternatives is very important.
If an alternative has an empty choice, it must be the last.
Empty choices in other positions will be reported as syntax errors.

For example to say that an \emph{atom} is an \emph{integer} or an \emph{expression in paranthesis}
you can write:
\begin{verbatimtab}[4]
    Atom -> integer | '\(' Expr '\)' ;
\end{verbatimtab}

\section{Repetitions}                                           \label{parser:repetitions}

Repetitions in grammar rules describe how many times an expression should be matched.

\begin{description}
    \item [A?] recognizes zero or one \emph{A}.
    \item [A*] recognizes zero or more \emph{A}.
    \item [A+] recognizes one or more \emph{A}.
    \item [A\{m,n\}] recognizes at least m and at most n \emph{A}.
\end{description}

Repetitions are greedy.
Repetitions are translated into Python loops.
Thus whatever the length of the repetitions, the Python stack will not overflow. 

\section{Precedence and grouping}

The figure~\ref{parser:precedence} lists the different structures in increasing precedence order.
To override the default precedence you can group expressions with parenthesis.

\begin{tableau}
\caption{Precedence in TPG expressions}                         \label{parser:precedence}
\begin{tabular}{| l | l |}
\hline
    Structure           & Example \\
\hline
\hline
    Alternative         & $A~\mid~B$ \\
\hline
    Sequence            & $A~B$ \\
\hline
    Repetitions         & $A?$, $A*$, $A+$ \\
\hline
    Symbol and grouping & $A$ and $(~\ldots~)$ \\
\hline
\end{tabular}
\end{tableau}

\section{Actions}

Grammar rules can contain actions as Python code.
Python code is copied verbatim into the generated code and
is delimited by \verb!$...$!, \verb!$...EOL!\footnote{EOL means End Of Line} or \verb!{{...}}!.

Please be aware that indentation should obey Python indentation rules.
See the grammar description for further information (see figure~\ref{grammar:indent}).

\subsection{Abstract syntax trees}                              \label{parser:AST}

An abstract syntax tree (AST) is an abstract representation of the structure of the input.
A node of an AST is a Python object (there is no constraint about its class).
AST nodes are completely defined by the user.

The figure~\ref{parser:ASTinst} shows a node symbolizing a couple.

\begin{code}
\caption{AST example}                                           \label{parser:ASTinst}
\begin{verbatimtab}[4]

class Couple:
    def __init__(self, a, b):
        self.a = a
        self.b = b

class Foo(tpg.Parser):
    r"""
    COUPLE/$Couple(a,b)$ -> '(' ITEM/a ',' ITEM/b ')' ;

    # which is equivalent to
    # COUPLE/c -> '(' ITEM/a ',' ITEM/b ')' $ c = Couple(a,b) $ ;
    """
\end{verbatimtab}
\end{code}

\subsubsection{Creating an AST}

AST are created in Python code (see section~\ref{parser:AST}).

\subsubsection{Updating an AST}

When parsing lists for example it is useful to save all the items of the list.
In that case one can use a list and its append method (see figure~\ref{parser:ASTadd}).

\begin{code}
\caption{AST update example}                                    \label{parser:ASTadd}
\begin{verbatimtab}[4]

class ListParser(tpg.Parser):
    r"""
    LIST/l ->
        '('                 $ l = []
            ITEM/a          $ l.append(a)
            ( ',' ITEM/a    $ l.append(a)
            )*
        ')'
        ;
    """
\end{verbatimtab}
\end{code}

\subsection{Text extraction}                                    \label{parser:mark}

TPG can extract a portion of the input string.
The idea is to put marks while parsing and then extract the text between the two marks.
This extracts the whole text between the marks, including the tokens defined as separators.

\subsection{Object}                                             \label{parser:object}

TPG 3 doesn't handle Python object as TPG 2.
Only identifiers, integers and strings are known.
Other objects must be written in Python code delimited either by \verb!$...$! or by \verb!{{...}}!.

\subsubsection{Argument lists and tuples}

Argument list is a comma separated list of objects.
\emph{Remember that arguments are enclosed in left and right angles.}

\begin{verbatimtab}[4]
    <object1, object2, object3>
\end{verbatimtab}

Argument lists and tuples have the same syntax except from the possibility to have
default arguments, argument lists and argument dictionnaries as arguments as in Python.

\begin{verbatimtab}[4]
    RULE<arg1, arg2=18, arg3=None, *other_args, **keywords> -> ;
\end{verbatimtab}

\subsubsection{Python code object}

A Python code object is a piece of Python code in double curly brackets or in dollar signs.
Python code used in an object expression must have only one line.

\begin{verbatimtab}[4]
    $ dict([ (x,x**2) for x in range(100) ]) $ # Python embeded in TPG
\end{verbatimtab}

\subsubsection{Text extraction}

Text extraction is done by the \emph{extract} method.
Marks can be put in the input string by the \emph{mark} method or the prefix \emph{\@} operator.

\begin{verbatimtab}[4]
    @beginning      # equivalent to $ beginning = self.mark()
    ...
    @end            # equivalent to $ end = self.mark()
    ...
    $ my_string = self.extract(beginning, end)
\end{verbatimtab}

\subsubsection{Getting the line and column number of a token}

The \emph{line} and \emph{column} methods return the line and column number of the current token.
If the first parameter is a mark (see~\ref{parser:mark}) the method returns the line number of the token following the mark.

\subsubsection{Backtracking}

The user can force the parser to backtrack in rule actions.
The module has a \emph{WrongToken} exception for that purpose (see figure~\ref{parser:wrongtoken}).

\begin{code}
\caption{Backtracking with \emph{WrongToken} example}           \label{parser:wrongtoken}
\begin{verbatimtab}[4]
    # NATURAL matches integers greater than 0
    NATURAL/n ->
        number/n
        $ if n<1: raise tpg.WrongToken $
        ;
\end{verbatimtab}
\end{code}

Parsers have another useful method named \emph{check} (see figure~\ref{parser:check}).
This method checks a condition.
If this condition is false then \emph{WrongToken} if called in order to backtrack.

\begin{code}
\caption{Backtracking with the \emph{check} method example}     \label{parser:check}
\begin{verbatimtab}[4]
    # NATURAL matches integers greater than 0
    NATURAL/n ->
        number/n
        $ self.check(n>=1) $
        ;
\end{verbatimtab}
\end{code}

A shortcut for the \emph{check} method is the \emph{check} keyword followed by the condition to check (see figure~\ref{parser:checkkw}).

\begin{code}
\caption{Backtracking with the \emph{check} keyword example}    \label{parser:checkkw}
\begin{verbatimtab}[4]
    # NATURAL matches integers greater than 0
    NATURAL/n ->
        number/n
        check $ n>=1 $
        ;
\end{verbatimtab}
\end{code}

\subsubsection{Error reporting}

The user can force the parser to stop and raise an exception.
The parser classes have a \emph{error} method for that purpose (see figure~\ref{parser:error_method}).
This method raises a \emph{SemanticError}.

\begin{code}
\caption{Error reporting the \emph{error} method example}   \label{parser:error_method}
\begin{verbatimtab}[4]
    # FRACT parses fractions
    FRACT/<n,d> ->
        number/n '/' number/d
        $ if d==0: self.error("Division by zero") $
        ;
\end{verbatimtab}
\end{code}

A shortcut for the \emph{error} method is the \emph{error} keyword followed by the object to give to the \emph{SemanticError} exception (see figure~\ref{parser:error_kw}).

\begin{code}
\caption{Error reporting the \emph{error} keyword example}  \label{parser:error_kw}
\begin{verbatimtab}[4]
    # FRACT parses fractions
    FRACT/<n,d> ->
        number/n '/' number/d
        ( check d | error "Division by zero" )
        ;
\end{verbatimtab}
\end{code}


    \chapter{Context sensitive lexer}                       \label{tpg:CSL}
        \section{Introduction}          \label{CSL}

Before the version 2 of TPG, lexers were context sensitive.
That means that the parser commands the lexer to match some tokens, i.e. different tokens can be matched in a same input string according to the grammar rules being used.
These lexers were very flexible but slower than context free lexers because TPG backtracking caused tokens to be matched several times.

In TPG 2, the lexer is called before the parser and produces a list of tokens from the input string.
This list is then given to the parser.
In this case when TPG backtracks the token list remains unchanged.

Since TPG 2.1.2, context sensitive lexers have been reintroduced in TPG.
By default lexers are context free but the \emph{CSL} option (see~\ref{grammar:lexer_option}) turns TPG into a context sensitive lexer.

\section{Grammar structure}

CSL grammar have the same structure than non CSL grammars (see~\ref{grammar:struct}) except from the \emph{lexer~=~CSL} option (see~\ref{grammar:lexer_option}).

\section{CSL lexers}

\subsection{Regular expression syntax}

The CSL lexer is based on the \emph{re} module.
The difference with non CSL lexers is that the given regular expression is compiled as this, without any encapsulation.

\subsection{Token matching}

In CSL parsers, tokens are matched as in non CSL parsers (see~\ref{lexer:token_matching}).

\section{CSL parsers}

There is no difference between CSL and non CSL parsers.

    \chapter{Debugging}
        \section{Introduction}          \label{debug}

When I need to debug a grammar I often add print statments to visualize the parser activity.
Now with TPG 3 it is possible to print such information automatically.

\section{Verbose parsers}

Normal parsers inherit from \emph{tpg.Parser}.
If you need a more verbose parser you can use \emph{tpg.VerboseParser} instead.
This parser prints information about the current token each time the lexer is called.
The debugging information has currently two level of details.

\begin{description}
    \item [Level 0] displays no information.
    \item [Level 1] displays tokens only when the current token matches the expected token.
    \item [Level 2] displays tokens if the current token matches or not the expected token.
\end{description}

The level is defined by the attribute \emph{verbose}. Its default value is 1.

\begin{code}
\caption{Verbose parser example}                            \label{debug:example}
\begin{verbatimtab}[4]
class Test(tpg.VerboseParser):
    r"""

    START -> 'x' 'y' 'z' ;

    """

    verbose = 2
\end{verbatimtab}
\end{code}

The information displayed by verbose parsers has the following format:
\begin{verbatim}
[eat counter][stack depth]callers: (line,row) <current token> == <expected token>
\end{verbatim}

\begin{description}
    \item [$eat counter$] is the number of calls of the lexer.
    \item [$stack depth$] is the depth of the Python stack since the axiom.
    \item [$callers$] is the list of non terminal symbols called before the current symbol.
    \item [$(line,row)$] is the position of the current token in the input string.
    \item [$==$] means the current token matches the expected token (level 1 or 2).
    \item [$!=$] means the current token doesn't match the expected token (level 2).
\end{description}


\cleardoublepage
\part{Some examples to illustrate TPG}                      \label{tpg:examples}
    \chapter{Complete interactive calculator}
        \section{Introduction}

This chapter presents an extention of the calculator described in the tutorial (see~\ref{tpg:tutorial}).
This calculator has more functions and a memory.

\section{New functions}

\subsection{Trigonometric and other functions}

This calculator can compute some numerical functions (\emph{sin}, \emph{cos}, \emph{sqrt}, \ldots).
The \emph{make\_op} function (see figure~\ref{tut:make_op}) has been extended to return these functions.
Tokens must also be defined to scan function names.
\emph{funct1} defines the name of unaries functions and \emph{funct2} defines the name of binaries functions.
Finally the grammar rule of the atoms has been added a branch to parse functions.
The \emph{Function} non terminal symbol parser unaries and binaries functions.

\subsection{Memories}

The calculator has memories.
A memory cell is identified by a name.
For example, if the user types \emph{$pi = 4*atan(1)$}, the memory cell named \emph{pi} will contain the value of \emph{$\pi$} and \emph{$cos(pi)$} will return \emph{$-1$}

To display the content of the whole memory, the user can type \emph{vars}.

The variables are saved in a dictionnary.
In fact the parser itself is a dictionnary (the parser inherits from the \emph{dict} class).

The \emph{START} symbol parses a variable creation or a single expression and the \emph{Atom} parses variable names (the \emph{Var} symbol parses a variable name and returns its value).

\section{Source code}

\subsection{TPG grammar}

The calculator source code can be a grammar for TPG.
I.e. the \emph{calc.g} file is translated into a \emph{calc.py} script by TPG.
Just type in:
\begin{verbatimtab}[4]
	tpg calc.g
\end{verbatimtab}

Here is the complete source code (\emph{calc.g}):

\verbatimtabinput[4]{../examples/calc/calc.g}

\subsection{Python script}

The calculator can be directly embeded in a Python script.
The grammar is in a string and compiled using the \emph{tpg} module.

Here is the complete source code (\emph{calc2.py}):

\verbatimtabinput[4]{../examples/calc/calc2.py}

    \chapter{Infix/Prefix/Postfix notation converter}
        \section{Introduction}

In the previous example, the parser computes the value of the expression on the fly, while parsing.
It is also possible to build an abstract syntax tree to store an abstract representation of the input.
This may be usefull when several passes are necessary.

This example shows how to parse an expression (infix, prefix or postfix) and convert it in infix, prefix and postfix notations.
The expression is saved in a tree. Each node of the tree corresponds to an operator in the expression. Each leave is a number.
Then to write the expression in infix, prefix or postfix notation, we just need to walk throught the tree in a particular order.

\section{Abstract syntax trees}

The AST of this converter has three types of node:

\begin{description}
    \item [class Op] is used to store operators (\verb$+$, \verb$-$, \verb$*$, \verb$/$, \verb$^$).
        It has two sons associated to the sub expressions.
    \item [class Atom] is an atomic expression (a number or a symbolic name).
    \item [class Func] is used to store functions.
\end{description}

These classes are instanciated by the \emph{\_\_init\_\_} method. The \emph{infix}, \emph{prefix} and \emph{postfix} methods return strings containing the representation of the node in \emph{infix}, \emph{prefix} and \emph{postfix} notation.

\section{Grammar}

\subsection{Infix expressions}

The grammar for infix expressions is similar to the grammar used in the previous example.

\begin{verbatimtab}[4]
EXPR/e -> TERM/e ( '[+-]'/op TERM/t $e=Op(op,e,t,1)$ )* ;
TERM/t -> FACT/t ( '[*/]'/op FACT/f $t=Op(op,t,f,2)$ )* ;
FACT/f -> ATOM/f ( '\^'/op FACT/e $f=Op(op,f,e,3)$ )? ;

ATOM/a -> ident/s $a=Atom(s)$ | '\(' EXPR/a '\)'
       |  func1/f '\(' EXPR/x            '\)' $a=Func(f,x)
       |  func2/f '\(' EXPR/x ',' EXPR/y '\)' $a=Func(f,x,y)
;
\end{verbatimtab}

\subsection{Prefix expressions}

The grammar for prefix expressions is very simple.
A compound prefix expression is an operator followed by two subexpressions.

\begin{verbatimtab}[4]
EXPR_PRE/e ->
    ident/s                             $ e=Atom(s)
|   '\(' EXPR_PRE/e '\)'
|   OP/<op,prec> EXPR_PRE/a EXPR_PRE/b  $ e=Op(op,a,b,prec)
|   func1/f EXPR/x                      $ e=Func(f,x)
|   func2/f EXPR/x EXPR/y               $ e=Func(f,x,y)

;

OP/<op,prec> ->
    '[+-]'/op   $ prec=1
|   '[*/]'/op   $ prec=2
|   '\^'/op     $ prec=3
;
\end{verbatimtab}

\subsection{Postfix expressions}

At first sight postfix and infix grammars may be very similar.
Only the position of the operators changes.
So a compound postfix expression is a first expression followed by a second and an operator.
This rule is left recursive.
As TPG is a descendant recursive parser, such rules are forbidden to avoid infinite recursion.
To remove the left recursion a classical solution is to rewrite the grammar like this:

\begin{verbatimtab}[4]
EXPR_POST/e -> ATOM_POST/a SEXPR_POST<a>/e ;

ATOM_POST/a ->
    ident/s                 $ a=Atom(s)
|   '\(' EXPR_POST/a '\)'
;

SEXPR_POST<e>/e ->
    EXPR_POST/e2
    (   OP/<op,prec> SEXPR_POST<$Op(op,e,e2,prec)$>/e
    |   func2/f SEXPR_POST<$Func(f, e, e2)$>/e
    )
|   func1/f SEXPR_POST<$Func(f, e)$>/e
|   ;
\end{verbatimtab}

The parser first searches for an atomic expression and then builds the AST by passing partial expressions by the attributes of the \emph{SEXPR\_POST} symbol.

\section{Source code}

Here is the complete source code (\emph{notation.py}):

\verbatimtabinput[4]{../examples/notation.pyg}


\end{document}
