\section{Introduction}

Before the version 2 of TPG, lexers were context sensitive.
That means that the parser commands the lexer to match some tokens, i.e. different tokens can be matched in a same input string according to the grammar rules being used.
These lexers were very flexible but slower than context free lexers because TPG backtracking caused tokens to be matched several times.

In TPG 2, the lexer is called before the parser and produces a list of tokens from the input string.
This list is then given to the parser.
In this case when TPG backtracks the token list remains unchanged.

Since TPG 2.1.2, context sensitive lexers have been reintroduced in TPG.
By default lexers are context free but the \emph{CSL} option (see~\ref{grammar:CSL}) turns TPG into a context sensitive lexer.

\section{Grammar structure}

CSL grammar have the same structure than non CSL grammars (see~\ref{grammar:struct}) except from the \emph{CSL} option (see~\ref{grammar:CSL}).

\section{CSL lexers}

\subsection{Regular expression syntax}

The CSL lexer is based on the \emph{re} module.
The difference with non CSL lexers is that the given regular expression is compiled as this, without any encapsulation.
Grouping is then possible and usable.

\subsection{Token definition}

In CSL lexers there is no predefined tokens.
Tokens are always inlined and there is no precedance issue since tokens are matched while parsing, when encountered in a grammar rule.

A token definition can be simulated by defining a rule to match a particular token (see figure~\ref{CSL:token_def}).

\begin{code}
\caption{Token definition in CSL parsers example}			\label{CSL:token_def}
\begin{verbatimtab}[4]
	number/int<n> -> '\d+'/n ;
\end{verbatimtab}
\end{code}

In non CSL parsers there are two kinds of tokens: true tokens and token separators.
To declare separators in CSL parsers you must use the special \emph{separator} rule.
This rule is implicitly used before matching a token.
It is thus necessary to distinguish lexical rules from grammar rules.
Lexical rule declarations start with the \emph{lex} keyword.
In such rules, the \emph{separator} rule is not called to avoid infinite recursion (\emph{separator} calling \emph{separator} calling \emph{separator} \ldots).
The figure~\ref{CSL:separator} shows a separator declaration with nested C++ like comments.

\begin{code}
\caption{Separator definition in CSL parsers examples}		\label{CSL:separator}
\begin{verbatimtab}[4]
	lex separator -> spaces | comment ;

	lex spaces -> '\s+' ;

	lex comment -> '/\*' in_comment* '\*/' ;        # C++ nested comments
	lex in_comment -> comment | '\*[^/]|[^\*]' ;
\end{verbatimtab}
\end{code}

\subsection{Token matching}

In CSL parsers, tokens are matched as in non CSL parsers (see~\ref{lexer:token_matching}).
There is a special feature in CSL parsers.
The user can benefit from the grouping possibilities of CSL parsers.
The text of the token can be saved with the infix \emph{/} operator.
The groups of the token can also be saved with the infix \emph{//} operator.
This operator (available only in CSL parsers) returns all the groups in a tuple.
For example, the figure~\ref{CSL:token_usage} shows how to read entire tokens and to split tokens.

\begin{code}
\caption{Token usage in CSL parsers examples}				\label{CSL:token_usage}
\begin{verbatimtab}[4]
	lex identifier/i -> '\w+'/s ;			# a single identifier

	lex string/s -> "'([^\']*)'"//<s> ;		# a string without the quotes

	lex item/<key,val> -> "(\w+)=(.*)"//<key,val> ;	# a tuple (key, value)
\end{verbatimtab}
\end{code}

\section{CSL parsers}

There is no difference between CSL and non CSL parsers except from lexical rules which look like grammar rules\footnote{In fact lexical rules and grammar rule are translated into Python in a very similar way}.
