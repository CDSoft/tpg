\section{General structure of the package}

TPG is delivered in a Python package named \emph{tpg}.
It is composed of:

\begin{description}
	\item [\_\_init\_\_.py]
		turns \emph{tpg} directory into a package.
		It defines some data about the current release (version, author, \ldots) and it imports in its local namespace the five useful objects \emph{compile}, \emph{translate}, \emph{LexerError}, \emph{ParserError} and \emph{SemanticError}.
	\item [base.py]
		defines the base class of the generated parsers and other classes used by these parsers. It's a kind of runtime for the parsers.
	\item [codegen.py]
		contains the classes used by the parser to represent the AST corresponding to the parsed grammar. Theses classes have the necessary methods for code generation.
	\item [parser.g]
		contains the grammar that recognizes TPG grammars. It defines the syntax of TPG grammars and builds the AST of the grammar.
	\item [parser.py]
		is automatically generated by TPG itself from \emph{parser.g}.
	\item [Release.py]
		contains release data (version, author, \ldots).
	\item [tpg]
		is a wrapper script for TPG.
		It reads a grammar and produces a Python script.
\end{description}
