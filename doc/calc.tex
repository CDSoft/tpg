\section{Introduction}

This chapter presents an extension of the calculator described in the tutorial (see~\ref{tpg:tutorial}).
This calculator has more functions and a memory.

\section{New functions}

\subsection{Trigonometric and other functions}

This calculator can compute some numerical functions (\emph{sin}, \emph{cos}, \emph{sqrt}, \ldots).
The \emph{make\_op} function (see figure~\ref{tut:make_op}) has been extended to return these functions.
Tokens must also be defined to scan function names.
\emph{funct1} defines the name of unary functions and \emph{funct2} defines the name of binary functions.
Finally the grammar rule of the atoms has been added a branch to parse functions.
The \emph{Function} non terminal symbol parser unary and binary functions.

\subsection{Memories}

The calculator has memories.
A memory cell is identified by a name.
For example, if the user types \emph{$pi = 4*atan(1)$}, the memory cell named \emph{pi} contains the value of \emph{$\pi$} and \emph{$cos(pi)$} returns \emph{$-1$}.

To display the content of the whole memory, the user can type \emph{vars}.

The variables are saved in a dictionnary.
In fact the parser itself is a dictionnary (the parser inherits from the \emph{dict} class).

The \emph{START} symbol parses a variable creation or a single expression and the \emph{Atom} parses variable names (the \emph{Var} symbol parses a variable name and returns its value).

\section{Source code}

Here is the complete source code (\emph{calc.pyg}):

\verbatimtabinput[4]{../examples/calc.pyg}
