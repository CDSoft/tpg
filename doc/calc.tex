\section{Introduction}

This chapter presents an extention of the calculator described in the tutorial (see~\ref{tpg:tutorial}).
This calculator has more functions and a memory.

\section{New functions}

\subsection{Trigonometric and other functions}

This calculator can compute some numerical functions (\emph{sin}, \emph{cos}, \emph{sqrt}, \ldots).
The \emph{make\_op} function (see figure~\ref{tut:make_op}) has been extended to return these functions.
Tokens must also be defined to scan function names.
\emph{funct1} defines the name of unaries functions and \emph{funct2} defines the name of binaries functions.
Finally the grammar rule of the atoms has been added a branch to parse functions.
The \emph{Function} non terminal symbol parser unaries and binaries functions.

\subsection{Memories}

The calculator has memories.
A memory cell is identified by a name.
For example, if the user types \emph{$pi = 4*atan(1)$}, the memory cell named \emph{pi} will contain the value of \emph{$\pi$} and \emph{$cos(pi)$} will return \emph{$-1$}.

To display the content of the whole memory, the user can type \emph{vars}.

The variables are saved in a dictionnary.
In fact the parser itself is a dictionnary (the parser inherits from the \emph{dict} class).

The \emph{START} symbol parses a variable creation or a single expression and the \emph{Atom} parses variable names (the \emph{Var} symbol parses a variable name and returns its value).

\section{Source code}

\subsection{TPG grammar}

The calculator source code can be a grammar for TPG.
I.e. the \emph{calc.g} file is translated into a \emph{calc.py} script by TPG.
Just type in:
\begin{verbatimtab}[4]
	tpg calc.g
\end{verbatimtab}

Here is the complete source code (\emph{calc.g}):

\verbatimtabinput[4]{../examples/calc/calc.g}

\subsection{Python script}

The calculator can be directly embeded in a Python script.
The grammar is in a string and compiled using the \emph{tpg} module.

Here is the complete source code (\emph{calc2.py}):

\verbatimtabinput[4]{../examples/calc/calc2.py}
