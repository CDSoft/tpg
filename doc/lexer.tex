\section{Regular expression syntax}

The lexer is based on the \emph{re}\footnote{\emph{re} is a standard Python module. It handles regular expressions. For further information about \emph{re} you can read \url{http://python.org/doc/2.2/lib/module-re.html}} module.
TPG profits from the power of Python regular expressions.
This document assumes the reader is familiar with regular expressions.

You can use the syntax of regular expressions as expected by the \emph{re} module except from the grouping syntax since it is used by TPG to decide which token is recognized.

\section{Token definition}                                  \label{lexer:token_def}

\subsection{Predefined tokens}

Tokens can be explicitely defined by the \emph{token} and \emph{separator} keywords.

A token is defined by:

\begin{description}
    \item [a name] which identifies the token.
        This name is used by the parser.
    \item [a regular expression] which describes what to match to recognize the token.
    \item [an action] which can translate the matched text into a Python object. It can be a function of one argument or a non callable object. It it is not callable, it will be returned for each token otherwise it will be applied to the text of the token and the result will be returned. This action is optional. By default the token text is returned.
\end{description}

Token definitions must end with a \emph{;} when no action is specified. The dots after the token name are optional.

See figure~\ref{lexer:tokens} for examples.

\begin{code}
\caption{Token definition examples}                         \label{lexer:tokens}
\begin{verbatimtab}[4]
    #     name     reg. exp        action
    token integer: '\d+'           int;
    token ident  : '[a-zA-Z]\w*'   ;
    token bool   : 'True|False'    $ lambda b: b=='True'

    separator spaces  : '\s+';     # white spaces
    separator comments: '#.*';     # comments
\end{verbatimtab}
\end{code}

The order of the declaration of the tokens is important. The first token that is matched is returned. The regular expression has a special treatment. If it describes a keyword, TPG also looks for a word boundary after the keyword. If you try to match the keywords \emph{if} and \emph{ifxyz} TPG will internally search \verb$if\b$ and \verb$ifxyz\b$. This way, \emph{if} won't match \emph{ifxyz} and won't interfere with general identifiers (\verb$\w+$ for example). This behaviour can be disabled since the verision 3 of TPG (see~\ref{grammar:word_boundary_option}).

There are two kinds of tokens. Tokens defined by the \emph{token} keyword are parsed by the parser and tokens defined by the \emph{separator} keyword are considered as separators (white spaces or comments for example) and are wiped out by the lexer.

\subsection{Inline tokens}

Tokens can also be defined on the fly. Their definition are then inlined in the grammar rules.
This feature may be useful for keywords or punctuation signs.
Inline tokens can not be transformed by an action as predefined tokens.
They always return the token in a string.

See figure~\ref{lexer:inline_tokens} for examples.

\begin{code}
\caption{Inline token definition examples}                  \label{lexer:inline_tokens}
\begin{verbatimtab}[4]
    IfThenElse ->
        'if' Cond
        'then' Statement
        'else' Statement
        ;
\end{verbatimtab}
\end{code}

Inline tokens have a higher precedence than predefined tokens to avoid conflicts (an inlined \emph{if} won't be matched as a predefined \emph{identifier}).

\section{Token matching}                                    \label{lexer:token_matching}

TPG works in two stages.
The lexer first splits the input string into a list of tokens and then the parser parses this list.
The default lexer is lazy in TPG 3. Tokens are generated while parsing.
This way TPG 3 need less memory when parsing huge files.

\subsection{Splitting the input string}

The lexer split the input string according to the token definitions (see~\ref{lexer:token_def}). When the input string can not be matched a \emph{tpg.LexicalError} exception is raised.

The lexer may loop indefinitely if a token can match an empty string since empty strings are everywhere.

\subsection{Matching tokens in grammar rules}

Tokens are matched as symbols are recognized.
Predefined tokens have the same syntax than non terminal symbols.
The token text (or the result of the function associated to the token) can be saved by the infix \emph{/} operator (see figure~\ref{lexer:token_ret_val}).

\begin{code}
\caption{Token usage examples}                              \label{lexer:token_ret_val}
\begin{verbatimtab}[4]
    S -> ident/i;
\end{verbatimtab}
\end{code}

Inline tokens have a similar syntax. You just write the regular expression (in a string). Its text can also be save (see figure~\ref{lexer:inline_token_ret_val}).

\begin{code}
\caption{Token usage examples}                              \label{lexer:inline_token_ret_val}
\begin{verbatimtab}[4]
    S -> '(' '\w+'/i ')';
\end{verbatimtab}
\end{code}

%\section{Special tokens}                                    \label{lexer:special_tokens}
%
%Their are some special tokens that have been requested by some users but these tokens can not be easily
%described by TPG using classical token definition (see~\ref{lexer:token_def}).
%
%\subsection{Indent and deindent tokens}                     \label{lexer:indent_deindent}
%
%TPG is written in Python so is should be easy to handle INDENT and DEDENT tokens as in Python language.
%These tokens are introduced in the source to be parsed by a preprocessor, before the lexer is activated.
%Spaces in the beginning the lines are replaced by indent and deindent tokens when needed.
%These special tokens are characters which ASCII codes are 16 and 17.
%These characters may not be used in regular text files.
%
%\subsubsection{Indent definition}
%
%The \emph{indent} option (see~\ref{grammar:options}) has been added to define the indentation.
%It has two values.
%The first one is a regular expression describing the indentation, usually spaces and tabulations.
%The second one is a regular expression describing the lines not to be taken into account, usually comments.
%This second parameter describes the beginning of the line, i.e. \emph{"\#"} will match lines starting
%with a \emph{\#}.
%
%\begin{code}
%\caption{Indent and deindent definition example}            \label{lexer:indent_definition}
%\begin{verbatimtab}[4]
%    set indent = "\s", "#"
%\end{verbatimtab}
%\end{code}
%
%\subsubsection{Indent and deindent usage}                   \label{lexer:indent_usage}
%
%When the \emph{indent} option is active, \emph{indent} and \emph{deindent} tokens are defined.
%They can be used as any other token.
%
%\begin{code}
%\caption{Indent and deindent example}            \label{lexer:indent_example}
%\begin{verbatimtab}[4]
%parser IndentParser:
%
%    set indent = "\s", "#"
%
%    BLOCK ->
%        (   INSTR
%        |   indent
%                BLOCK
%            deindent
%        )*;
%\end{verbatimtab}
%\end{code}
